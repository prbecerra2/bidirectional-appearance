\chapter{Étude de la condition d'apparition du régime bidirectionnel} \label{ch:apparition}

\localtableofcontents

\vskip 4\baselineskip
\noindent\textbf{\large Résumé}
\vskip 0pt
\noindent\rule{\linewidth}{.2pt}
\vskip 6pt
\noindent Ce chapitre présente une étude expérimentale conduite pour estimer les conditions d'apparition d'un écoulement bidirectionnel au niveau de l'exutoire à partir d'un régime monodirectionnel. Cette étude est axée sur l'analyse des variations de température localisées dans la section de l'orifice et sur l'observation de l'écoulement pour estimer la zone de transition durant les expériences. Une relation donnant les surfaces d'ouvertures est proposée.
\vskip 0pt
\noindent\rule{\linewidth}{.2pt}

\newpage

\section{Introduction}
Ce chapitre présente l'influence de la surface de l'amenée d'air en partie basse $a_B$ sur le régime de l'écoulement au niveau de l'exutoire de section $a_T$, afin d'estimer les critères de transition entre les deux régimes. La section qui correspond à la zone de transition ou critique sur $a_B$ est définie par $a^{crit}$.

Deux méthodes sont retenues. La première repose sur l'analyse de signaux de température au niveau de l'orifice. La seconde s’appuie sur des visualisations avec un plan laser. L'objectif de cette étude est de caractériser les changements de l'écoulement entre le régime monodirectionnel et bidirectionnel.

Le tableau \ref{tab:essais_resist} résume les différentes configurations réalisées sur le dispositif pour estimer la zone de transition de régime. Le diamètre $D$ correspond au diamètre de l'orifice horizontal ou trémie. La puissance nominale de la source est déterminée par le nombre de résistances électriques de puissance nominale électrique $P_e$ de \SI{2}{kW}. Les puissances intermédiaires sont obtenues par la modulation de la résistance inférieure avec un variateur.

\begin{table}[H]
\centering
\caption{Récapitulatif des essais réalisés avec résistances électriques pour identifier la transition de régime.}
\setlength\extrarowheight{0.5pt}
\begin{tabular}{c c c}
\hline
$D\,(\si{mm})$ & $P_e\,(\si{kW})$ & Nombre d'essais \rule{0pt}{14pt}\\[0.5ex]
\hline
76.2 & 1.2 & 2 \\
76.2 & 2.0 & 3 \\
76.2 & 4.0 & 3 \\
\midrule[0.1pt]  
127.0 & 1.2 & 2 \\ 
127.0 & 2.0 & 3 \\
127.0 & 4.0 & 3 \\
\midrule[0.1pt]  
152.4 & 2.0 & 6 \\ 
152.4 & 4.0 & 6 \\ 
\midrule[0.1pt]  
190.5 & 1.2 & 2 \\ 
190.5 & 2.0 & 6 \\ 
190.5 & 3.4 & 2 \\
190.5 & 4.0 & 6 \\  
\hline
\end{tabular}
\label{tab:essais_resist}
\end{table}

\section{Étude des signaux de température}
Dans un premier temps on va illustrer la réponse des thermocouples placés dans l'épaisseur de l'évent pour les régimes monodirectionnel et bidirectionnel.

La figure \ref{fig:tc_mono_D191_2R} présente l'évolution des températures à la trémie en fonction du rapport des surfaces des ouvertures $R=a_B/a_T$ à partir d'une condition de régime monodirectionnel vérifiée visuellement et une condition de régime thermiquement établi avec une source de puissance électrique de \SI{4}{kW}. Le régime établi est atteint après la phase de mise en chauffe du dispositif d'environ \SI{4}{hr}. Le temps de chaque intervalle du rapport $R$ correspond à \SI{20}{min}, temps minimum estimé suite à plusieurs essais pour atteindre un état quasi-stationnaire après un changement de la configuration de ventilation du local. On observe une superposition des signaux des huit thermocouples au début de ce régime où la fluctuation par rapport à la moyenne de chaque intervalle ne dépasse de \SI{2}{\celsius}. La température moyenne dans chaque intervalle augmente au fur à mesure que l'on réduit la section d'entrée.

\begin{figure}
\centering
\resizebox{0.69\textwidth}{!}{\input{figures/T_tremie_ER28_D191_2R_mono-fig}}
\caption{Évolution des températures à la trémie en fonction du temps et du rapport des surfaces des ouvertures $R=a_B/a_T$ pour un diamètre $D=\SI{191}{mm}$ en régime monodirectionnel. La section $a_B$ est modifiée toutes les \SI{20}{min}}
\label{fig:tc_mono_D191_2R}
\end{figure}

Les températures au niveau de la trémie pour une condition de confinement total ($R=0$), sont présentées sur la figure \ref{fig:tc_bidi_D191_2R} à la fin de cet essai. Les fluctuations par rapport à la moyenne sont environ 5 fois plus grande que pour un régime monodirectionnel avec une grande ouverture en partie basse. L'augmentation des fluctuations des températures au niveau de l'exutoire lié au confinement du local, indique que l'identification de la transition de régime peut être obtenue à partir des fluctuations de température.

\begin{figure}
\centering
\resizebox{0.69\textwidth}{!}{\input{figures/T_tremie_ER28_D191_2R_bidi-fig}}
\caption{Évolution des températures à la trémie en fonction du temps pour un rapport des surfaces $R=0$ et un diamètre $D=\SI{191}{mm}$ en régime bidirectionnel.}
\label{fig:tc_bidi_D191_2R}
\end{figure}

Afin d'étudier la réponse d'un thermocouple, l'évolution temporelle de la température mesurée sur un essai complet est présentée sur la figure \ref{fig:tc_est_2R_trans}. Cet essai commence pour une condition ouverte en régime monodirectionnel jusqu'au confinement total. On remarque globalement l'augmentation des fluctuations avec le degré de confinement et une augmentation progressive de la température moyenne dans chaque intervalle sauf pour la condition fermée où les débits échangés s'équilibrent avec un refroidissement local près de l'orifice.

Particulièrement sur cet essai, on pourrait estimer à partir des fluctuations que la transition de régime commence pour le rapport $R=0.19$ (indiquée par le signe d'interrogation), où des fluctuations ponctuelles apparaissent et doublent celles du régime monodirectionnel. Cependant, c'est dans l'intervalle suivant de $R=0.14$, que la fréquence des fluctuations ponctuelles augmente ainsi que leurs amplitudes, confirmant l'existence d'un écoulement bidirectionnel. Cet écoulement bidirectionnel a été vérifié visuellement en éclairant le flux d'échange avec un laser depuis cette intervalle de $R=0.14$ jusqu'à la fin de l'essai. 

\begin{figure}
\centering
\resizebox{0.75\textwidth}{!}{\input{figures/T_E_ER28_D191_2R_trans-fig}}
\caption{Évolution de la température du thermocouple TC-E de la trémie en fonction du temps et du rapport des surfaces des ouvertures $R$ pour un diamètre $D=\SI{191}{mm}$ et une puissance $P_e=\SI{4}{kW}$, illustrant le passage d'un régime d'écoulement à l'autre.}
\label{fig:tc_est_2R_trans}
\end{figure}

Utilisant les mêmes puissance électrique et diamètre de trémie, deux essais ont été réalisés en variant le sens de la ventilation du local, c'est à dire, commençant par un régime bidirectionnel en $R=0$ vers un écoulement monodirectionnel et réciproquement. L'objectif de ces essais c'est d'étudier l'influence du sens de la ventilation, soit en ouvrant ou fermant la section $a_B$, sur la transition du régime. L'évolution des températures mesurées par la thermocouple TC-E est illustré sur la figure \ref{fig:tc_est_2R_hyst}. On remarque que les températures à la trémie sont supérieures dans l'essai commençant par un régime bidirectionnel. La transition à partir d'un régime monodirectionnel a été observée visuellement pour un rapport $R= 0.13$, et dans l'autre essai partant d'un régime bidirectionnel, la transition s'est présentée en $R=0.14$. Cette légère différence peut être liée à l'augmentation de l'énergie emmagasinée dans l'enceinte au cours de l'essai partant d'un écoulement bidirectionnel et le besoin d'une ouverture en partie basse supérieure pour basculer au régime monodirectionnel.

À partir des signaux de ces deux essais, il est très difficile d'identifier la transition, marquée par le signe d'interrogation sur les figures, dont l'effet du sens n'est pas très significatif et reste dans l'incertitude des mesures.

\begin{figure}
\centering
\begin{subfigure}[t]{0.485\textwidth}
  \centering
  \resizebox{\textwidth}{!}{\input{figures/T_E_ER31_D191_2R_hyst-fig}}
  \caption{}
  \label{fig:hyst_mono}
\end{subfigure}
\begin{subfigure}[t]{0.485\textwidth}
  \centering
  \resizebox{\textwidth}{!}{\input{figures/T_E_ER32_D191_2R_hyst-fig}}
  \caption{}
  \label{fig:hyst_mono}
\end{subfigure}
\caption{Évolution de la température du thermocouple TC-E de la trémie en fonction du temps et du rapport des surfaces des ouvertures $R$ pour un diamètre ${D=\SI{191}{mm}}$ et une puissance $P_e=\SI{4}{kW}$. Effet du sens dans l'obtention de la section critique $a^{crit}_B$ partant d'un écoulement monodirectionnel en (a) et bidirectionnel en (b).}
\label{fig:tc_est_2R_hyst}
\end{figure}

Dans la suite de cette étude, on va présenter deux approches pour estimer quel type de seuil pourrait-on utiliser pour identifier la transition basée sur les signaux des températures situées dans la section de l'orifice.

\subsection{Étude de la température moyenne}
Cette section présente une étude sur la température moyenne à la trémie comme critère permettant d'identifier la transition.

Les températures moyennes de chaque thermocouple sur chaque intervalle ainsi que la moyenne de l'ensemble pour la même condition de ventilation sont présentées sur la figure \ref{fig:tc_moy_D191_2R}.

Un écartement des températures moyennes de chaque thermocouple par rapport à la moyenne de l'ensemble s'observe dès qu'on commence à restreindre l'entrée d'air en partie basse. Pour cet essai, présenté en figure \ref{fig:tc_est_2R_trans}, l'écartement commence à partir de $R=0.19$ mais l'estimation de la transition de régime à partir du degré d'écartement n'est pas suffisant ni évident. Dans l'intervalle suivant ($R=0.14$) l'écartement entre les signaux est plus marqué, dont la moyenne de l'ensemble est proche du maximum observé en $R=0.08$. Sur ces deux dernières configurations, l'écoulement est bidirectionnel et a été confirmé visuellement. La transition dans cet essai est comprise entre les valeurs $0.14<R<0.19$.

D'autre part, pour la condition complètement fermée, l'écart entre les thermocouples est plus clair, et notamment on remarque une différence de plus de \SI{12}{\celsius} entre les thermocouples O et NE séparées de moins de \SI{15}{cm}. Cette différence en moyenne entre les thermocouples pourrait être un indicateur de la préférence spatiale des fluides montant ou descendant à travers la trémie en régime bidirectionnel.

\begin{figure}
\centering
\resizebox{0.67\textwidth}{!}{\input{figures/TC_tremie_moy_ER28_D191_2R-fig}}
\caption{Évolution des températures moyennes à la trémie en fonction du rapport $R$ pour un diamètre $D=\SI{191}{mm}$ et une puissance $P_e=\SI{4}{kW}$.}
\label{fig:tc_moy_D191_2R}
\end{figure}

La figure \ref{fig:tc_moy_D191_1R} présente les températures moyennes pour un diamètre $D=\SI{191}{mm}$ mais avec une résistance électrique de puissance $P_e=\SI{2}{kW}$. Le comportement des températures moyennes est similaire à celui obtenu pour une puissance $P_e={4}{kW}$. La température moyenne augmente avec le confinement et décroit dès que l'écoulement descendant est présent pour refroidir localement l'orifice jusq'au confinement total. Cependant l'écartement de la moyenne de chaque thermocouple sur la moyenne de l'ensemble commence à un rapport $R=0.27$, supérieur au cas précédent. La détermination de la transition ne peut pas être confirmée en s'appuyant que sur l'écartement des températures relatif à la moyenne de l'ensemble.

\begin{figure}
\centering
\resizebox{0.67\textwidth}{!}{\input{figures/TC_tremie_moy_ER30_D191_1R-fig}}
\caption{Évolution des températures moyennes à la trémie en fonction du rapport $R$ pour un diamètre $D=\SI{191}{mm}$ et une puissance $P_e=\SI{2}{kW}$.}
\label{fig:tc_moy_D191_1R}
\end{figure}

Si on change le diamètre de la trémie à $D=\SI{152}{mm}$ on observe la même tendance sur les températures moyennes décrite précédemment. La figure \ref{fig:tc_moy_D152_1R} illustre le début de l'écartement des thermocouples à un $R=0.21$, dont la transition a été observée visuellement en $R=0.13$. On confirme que l’identification de la transition par l'écartement des températures par rapport à l'ensemble reste imprécise.

Le fait de réduire le diamètre de la trémie ou d'augmenter la puissance de la source de chaleur décale la transition de régime vers rapport $R$ inférieur, c'est à dire qu'il faut réduire la surface d'amenée d'air pour faire apparaitre un écoulement bidirectionnel suite à ces variations. Ce constat a été vérifié visuellement avec d'autres diamètres de trémie et d'autres puissances électriques. En résumé, le changement de régime de l'écoulement dépend de la température dans l'enceinte ainsi que de la section de l'ouvrant. 

\begin{figure}
\centering
\resizebox{0.6\textwidth}{!}{\input{figures/TC_tremie_moy_ER23_D152_1R-fig}}
\caption{Évolution des températures moyennes à la trémie en fonction du rapport $R$ pour un diamètre $D=\SI{152}{mm}$ et une puissance $P_e=\SI{2}{kW}$.}
\label{fig:tc_moy_D152_1R}
\end{figure}

L'effet de la puissance de la source est montré sur la figure \ref{fig:T_moy_effet_R} indiquant que l'augmentation de la température décale la section critique, proche de la valeur maximale, vers une valeur inférieure de $R$.

\begin{figure}
\centering
\begin{subfigure}[t]{0.485\textwidth}
  \centering
  \resizebox{\textwidth}{!}{\input{figures/T_tremie_moy_effet_R_D191-fig}}
  \caption{}
  \label{fig:T_moy_effet_R_D191}
\end{subfigure}
\begin{subfigure}[t]{0.485\textwidth}
  \centering
  \resizebox{\textwidth}{!}{\input{figures/T_tremie_moy_effet_R_D152-fig}}
  \caption{}
  \label{fig:T_moy_effet_R_D152}
\end{subfigure}
\caption{Évolution de la température moyenne en fonction du rapport des surfaces des ouvertures $R$. Effet de la puissance de la source pour un diamètre ${D=\SI{191}{mm}}$ en (a) et ${D=\SI{152}{mm}}$ en (b).}
\label{fig:T_moy_effet_R}
\end{figure}

L'effet du diamètre est illustré sur la figure \ref{fig:T_moy_effet_D} où les températures en confinement total sont plus grandes pour le diamètre le plus petit. La différence entre les ouvrants de diamètre ${D=\SI{127}{mm}}$ et ${D=\SI{152}{mm}}$ n'est pas significative dans la zone critique proche des maximums. L'effet de la température est plus sensible que la section de l'ouvrant.

\begin{figure}
\centering
\begin{subfigure}[t]{0.485\textwidth}
  \centering
  \resizebox{\textwidth}{!}{\input{figures/T_tremie_moy_effet_D_1R-fig}}
  \caption{}
  \label{fig:T_moy_effet_D_1R}
\end{subfigure}
\begin{subfigure}[t]{0.485\textwidth}
  \centering
  \resizebox{\textwidth}{!}{\input{figures/T_tremie_moy_effet_D_2R-fig}}
  \caption{}
  \label{fig:T_moy_effet_D_2R}
\end{subfigure}
\caption{Évolution de la température moyenne en fonction du rapport des surfaces des ouvertures $R$. Effet du diamètre de l'ouvrant pour une puissance électrique ${P_e=\SI{2}{kW}}$ en (a) et ${P_e=\SI{4}{kW}}$ en (b).}
\label{fig:T_moy_effet_D}
\end{figure}

\subsection{Étude des fluctuations des températures}
Une étude sur les écarts type de chaque thermocouple de la trémie a été menée sur l'ensemble des essais pour déterminer si ce paramètre qui prend en compte les fluctuations, pourrait être un meilleur indicateur que la température moyenne. L'écart type pour chaque thermocouple a été calculé selon la formule \eqref{eq:sigma}:

\begin{equation}
\sigma = \sqrt{\frac{1}{n} \sum\limits_{i=1}^{n} (x_i - \bar x)^2 } \quad ,
\label{eq:sigma}
\end{equation}
\noindent où $\bar x$ est la moyenne de $n$ événements déterminée par $\bar x = \frac{1}{n} \sum\limits_{i=1}^{n} x_i$.
\vspace{0.5\baselineskip}

Dans une première partie, une analyse qualitative est abordée en utilisant comme cas d'étude l'essai avec un diamètre de trémie $D=\SI{191}{mm}$ et deux résistances. La figure \ref{fig:sigma_tc_D191_2R} présente l'évolution des écarts types en fonction du rapport des ouvertures $R$. La fréquence d'acquisition est de \SI{5}{Hz} et le temps des intervalles est de \SI{20}{min}.

L'écart type de chaque thermocouple dans les intervalles ${R>0.20}$ est proche de la moyenne de l'ensemble d'environ \SI{1}{\celsius}. On pourrait estimer pour cet essai que si ${R>0.20}$, alors le régime est monodirectionnel. C'est à partir de ${R\leq 0.20}$ qu'on observe un écartement entre les écarts type indiquant un changement sur l'écoulement au sein de la trémie lié au fluide descendant plus froid qui entre dans le local.

Au sein de la trémie, les réponses à la présence d'un écoulement descendant entre les thermocouples TC-E et TC-O et TC-NO, en couleur rouge, jaune et marron respectivement, s'opposent. Cette différence est associée à la position relative de ces thermocouples par rapport à la source. La thermocouple TC-O est positionné de coté de la source et son écart type par rapport à la moyenne de l'ensemble est toujours inférieur sauf quand $R=0$ en condition fermée. À ce point les différences de température du coté Ouest et Nord-ouest sont plus importantes que du coté Est dont l'écart type est le plus bas de l'ensemble. Cependant les thermocouples TC-E et TC-NE présentent l'écart type plus grand pour la plupart des essais dès qu'on observe un régime bidirectionnel et pourrait indiquer une préférence du fluide descendant à rentrer par cette zone.

\begin{figure}
\centering
\resizebox{0.7\textwidth}{!}{\input{figures/sigma_T_tremie_ER28_D191_2R-fig}}
\caption{Évolution des écarts types des températures à la trémie en fonction du rapport $R=a_B/a_T$ pour un diamètre $D=\SI{191}{mm}$ et une puissance $P_e=\SI{4}{kW}$.}
\label{fig:sigma_tc_D191_2R}
\end{figure}

Sur la figure \ref{fig:sigma_tc_D191_D152_1R}, on observe une organisation similaire des thermocouples par rapport à la moyenne des écarts types de l'ensemble, représentée par des carrés sur la figure. Le thermocouple TC-E est celui qui a l'écart type le plus grand dès que l'écartement se révèle, en opposition au TC-O, coté source, qui montre un écart type plus faible jusqu'avant le confinement total. En revanche la TC-NO présente l'écart type le plus grand en confinement total et le TC-E inverse sa tendance et devient le thermocouple avec l'écart type le plus bas.

\begin{figure}
\centering
\begin{subfigure}[t]{0.495\textwidth}
\resizebox{\textwidth}{!}{\input{figures/sigma_T_tremie_ER30_D191_1R-fig}}
\caption{}
\label{fig:sigma_tc_D191_1R}
\end{subfigure}
\begin{subfigure}[t]{0.495\textwidth}
\centering
\resizebox{\textwidth}{!}{\input{figures/sigma_T_tremie_ER27_D152_1R-fig}}
\caption{}
\label{fig:sigma_tc_D152_1R}
\end{subfigure}
\caption{Évolution des écarts types des températures à la trémie en fonction du rapport des surfaces des ouvertures $R=a_B/a_T$ pour un diamètre (a) $D=\SI{191}{mm}$ et (b) $D=\SI{152}{mm}$ et une puissance $P_e=\SI{2}{kW}$.}
\label{fig:sigma_tc_D191_D152_1R}
\end{figure}

Les thermocouples TC-E et TC-NE présentent des écarts types plus grands comparés à ces voisins au début du régime bidirectionnel. Ce constat peut nous indiquer que ce sont les plus réactifs aux premières bouffées d'air froid qui entrent à la trémie. On va analyser plus particulièrement les réponses de ses deux thermocouples au changement de la surface d'amenée d'air pour estimer le passage au régime bidirectionnel.

Afin de réaliser une analyse quantitative pour prendre en compte les fluctuations, on va d'abord calculer une moyenne glissante ou mobile. Cette moyenne est appliquée sur 5 périodes avant et après sur chaque point, dont la fréquence d'acquisition de \SI{5}{Hz}, pour ne pas perdre beaucoup d'informations. Comme référence, la fréquence caractéristique d'un écoulement bidirectionnel en confinement total a été estimée numériquement inférieur à \SI{1}{Hz} par \textcite{harrison_spall_2003}. Ensuite on va soustraire cette moyenne glissante à la température brute de chaque thermocouple pour représenter l'évolution des fluctuations centrées sur l'axe des abscisses.

La figure \ref{fig:tc_gliss_est_2R} montre l'évolution des fluctuations de la température enregistrée par le thermocouple TC-E utilisant la procédure indiquée précédemment. Les fluctuations en confinement total dépassent \SI{12}{\celsius} pour cet essai, dont les fluctuations en régime pleinement monodirectionnel sont inférieures à \SI{2}{\celsius}.

\begin{figure}
\centering
\resizebox{0.8\textwidth}{!}{\input{figures/T_gliss_E_ER28_D191_2R-fig}}
\caption{Évolution des fluctuations de la température du TC-E en fonction du temps et du rapport $R=a_B/a_T$ pour un diamètre $D=\SI{191}{mm}$ et une puissance $P_e=\SI{4}{kW}$.}
\label{fig:tc_gliss_est_2R}
\end{figure}

Pour chaque diamètre de trémie et chaque puissance électrique utilisée, l'écart type des fluctuations brutes des thermocouples a été déterminé pour la surface d'amenée d'air en partie basse maximal en régime monodirectionnel, dont le rapport des ouvertures est compris entre $1.10 \leq R < 1.40$. Le tableau \ref{tab:ecart_type_moyen} résume les écarts types moyens selon la puissance électrique où la variation de l'écart type est plus sensible qu'au changement de diamètre d'orifice. Pour une même puissance la variation de l'écart type est inférieur à 10\% entre les quatre diamètres utilisés représentée par l'incertitude dans ce tableau.

\begin{table}[H]
\centering
\caption{Écart type moyen en fonction de la puissance de la source en régime monodirectionnel pour un rapport d'ouvertures $R \geq 1.10$.}
\setlength\extrarowheight{0.5pt}
\begin{tabular}{c c c}
\hline
$\sigma\,(\si{\celsius})$ & $P_e\,(\si{kW})$ \rule{0pt}{14pt}\\[0.5ex]
\hline
$0.30 \pm 0.02$ & 1.2 \\ 
$0.35 \pm 0.03$ & 2.0 \\ 
$0.50 \pm 0.03$ & 3.4 \\
$0.60 \pm 0.03$ & 4.0 \\  
\hline
\end{tabular}
\label{tab:ecart_type_moyen}
\end{table}

La règle de $3\sigma$ ou de trois écart type est utilisé en statistique pour indiquer que 99.7\% des valeurs se situent dans un intervalle centré autour de sa moyenne suivant une loi normale de distribution. Sous l'hypothèse d'une loi normale, cette règle est aussi utilisée pour déterminer les valeurs aberrantes d'un échantillon s'ils sont éloignés de plus de trois écart type de la moyenne.

Si on suppose que les événements d'un régime bidirectionnel sont des valeurs aberrantes par rapport à un écoulement monodirectionnel établi, ces événements devraient être supérieurs avec la valeur de $3\sigma$. Pour chaque essai les événements supérieurs à $3\sigma$ utilisant les valeurs d'écart type du tableau \ref{tab:ecart_type_moyen} ont été comptabilisés. 

La figure \ref{fig:tc_gliss_est_2R_3_sigma} montre les événements supérieurs à $3\sigma$ qu'apparaissent à partir d'une valeur inférieure de $R=0.19$ pour cet essai avec un diamètre $D=\SI{191}{mm}$ et une puissance de \SI{4}{kW}. Sous le régime bidirectionnel presque tous les événements sont supérieurs à cette valeur seuil. Cette règle peut nous indiquer l'apparition du régime bidirectionnel dès que les écarts types des fluctuations dépassent la valeur de $3\sigma$ de l’écart type du régime monodirectionnel pour une puissance de la source donnée.

\begin{figure}
\centering
\resizebox{0.725\textwidth}{!}{\input{figures/T_gliss_E_ER28_D191_2R_3_sigma-fig}}
\caption{Évolution des fluctuations de la température du TC-E en fonction du temps et du rapport $R=a_B/a_T$ pour un diamètre $D=\SI{191}{mm}$ et une puissance $P_e=\SI{4}{kW}$ avec une zone d'analyse utilisant un écart type de $\pm 3\,\sigma$.}
\label{fig:tc_gliss_est_2R_3_sigma}
\end{figure}

Le nombre des événements qui dépassent la valeur de $3\sigma$ sont exposés dans le tableau \ref{tab:event_3_sigma_D191_2R} pour les huit thermocouples de la trémie en fonction du rapport $R$ et pour le même essai de la figure précédente.

\begin{table}[H]
\centering
\caption{Nombre d'événements supérieurs à $3\sigma$ de chaque thermocouple de la trémie pour un essai avec un diamètre $D=\SI{191}{mm}$ et une puissance $P_e=\SI{4}{kW}$.}
%\setlength\extrarowheight{0.5pt}
\begin{tabular}{c c c c c c c c c}
\hline
$R=a_B/a_T$ & TC-E & TC-NE & TC-SE & TC-S & TC-SO & TC-O & TC-NO & TC-N \rule{0pt}{14pt}\\[0.5ex]
\hline
1.40 & 0 & 0 & 8 & 2 & 46 & 33 & 19 & 26 \\
0.87 & 0 & 0 & 0 & 0 & 0 & 0 & 6 & 48 \\
0.52 & 0 & 0 & 0 & 0 & 0 & 0 & 1 & 28 \\
0.35 & 0 & 0 & 0 & 0 & 0 & 0 & 0 & 0 \\
0.26 & 9 & 12 & 0 & 7 & 11 & 0 & 15 & 15 \\
0.19 & 330 & 355 & 119 & 99 & 130 & 35 & 246 & 406 \\
0.14 & 2267 & 2224 & 1788 & 1352 & 1800 & 1072 & 2374 & 2615 \\
0.08 & 3429 & 3291 & 3486 & 3586 & 3795 & 3389 & 4073 & 3989 \\
0 & 4159 & 4334 & 4181 & 4146 & 4411 & 4252 & 4548 & 4669 \\
\hline
\end{tabular}
\label{tab:event_3_sigma_D191_2R}
\end{table}

Suite à la comptabilisation des événements supérieurs à $3\sigma$, le critère proposé pour estimer le début de la transition est la fréquence moyenne sur le temps de chaque intervalle de ces événements. Ce critère est fixé à une fréquence minimale de $f=\SI{0.20}{Hz}$ à partir d'une étude réalisée sur les thermocouples TC-E et TC-NE. Ces thermocouples sont considérés par leurs positions relatives à la source et l'écoulement, être les plus réactives à un changement de régime. Le tableau \ref{tab:freq_3_sigma_D191_2R} montre les fréquences associées aux événements présentés dans le tableau précédent. Particulièrement dans cet essai, le critère est satisfait à partir d'un rapport $R=0.19$, de même pour les thermocouples TC-NO et TC-N. On remarque aussi l'augmentation de la fréquence à partir de $R=0.08$ confirmant un écoulement bidirectionnel et quand $R=0$, les fréquences sont du même ordre de grandeur. 

\begin{table}[H]
\centering
\caption{Fréquence des événements supérieures à $3\sigma$ de chaque thermocouple de la trémie pour un essai avec un diamètre $D=\SI{191}{mm}$ et une puissance $P_e=\SI{4}{kW}$.}
%\setlength\extrarowheight{0.5pt}
\begin{tabular}{c c c c c c c c c}
\hline
$R=a_B/a_T$ & TC-E & TC-NE & TC-SE & TC-S & TC-SO & TC-O & TC-NO & TC-N \rule{0pt}{14pt}\\[0.5ex]
\hline
1.40 & 0 & 0 & 0.01 & 0 & 0.04 & 0.03 & 0.02 & 0.02 \\
0.87 & 0 & 0 & 0 & 0 & 0 & 0 & 0.01 & 0.04 \\
0.52 & 0 & 0 & 0 & 0 & 0 & 0 & 0 & 0.02 \\
0.35 & 0 & 0 & 0 & 0 & 0 & 0 & 0 & 0 \\
0.26 & 0.01 & 0.01 & 0 & 0.01 & 0.01 & 0 & 0.01 & 0.01 \\
0.19 & 0.28 & 0.30 & 0.10 & 0.08 & 0.11 & 0.03 & 0.21 & 0.34 \\
0.14 & 1.89 & 1.85 & 1.49 & 1.13 & 1.50 & 0.89 & 1.98 & 2.18 \\
0.08 & 2.86 & 2.74 & 2.91 & 2.99 & 3.16 & 2.82 & 3.39 & 3.32 \\
0 & 3.47 & 3.61 & 3.48 & 3.46 & 3.68 & 3.54 & 3.79 & 3.89 \\
\hline
\end{tabular}
\label{tab:freq_3_sigma_D191_2R}
\end{table}

L'analyse utilisant la règle de $3\sigma$ est maintenant présentée pour un essai avec un diamètre de trémie $D=\SI{127}{mm}$ et une puissance de \SI{4}{kW} et un pas de temps entre intervalles de \SI{12}{min}. Ces intervalles montrent qu'il faut un temps pour que le régime bidirectionnel se mette en place associé au rapport d'ouvertures $R$ et à la température à la trémie. Dans ce cas la transition est observée pour un $R=0.07$ suivant la critère proposé. D'autre part dans l'intervalle précédent de $R=0.10$ les fréquences des fluctuations sont de l'ordre de \SI{0.10}{Hz} en-dessus du critère limite, cependant visuellement l'écoulement est observé bidirectionnel dans cette période jusqu'au confinement en $R=0$ où les fréquences sont de l'ordre de \SI{0.70}{Hz}. Par contre pour d'autres essais, ces fréquences sont supérieures à \SI{3}{Hz}. Ceci implique que le critère doit être accompagné d'une vérification visuelle de l'écoulement et d'un temps minimum entre intervalles, estimé à \SI{20}{min}. Ce temps a été optimisé pour examiner différents configurations de façon progressive dans la journée.

\begin{figure}
\centering
\resizebox{0.725\textwidth}{!}{\input{figures/T_gliss_E_ER11_D127_2R_3_sigma-fig}}
\caption{Évolution des fluctuations de la température du TC-E en fonction du temps et du rapport $R=a_B/a_T$ pour un diamètre $D=\SI{127}{mm}$ et une puissance $P_e=\SI{4}{kW}$ avec une zone d'analyse utilisant un écart type de $\pm 3\,\sigma$.}
\label{fig:tc_gliss_est_D127_2R_3_sigma}
\end{figure}

Le tableau \ref{tab:result_bidi_tc} synthétise les résultats obtenus pour estimer le début de la transition de régime avec la règle de $3\sigma$ sur les écarts types des thermocouples à la trémie. Les valeurs de la section critique $a^{crit}_B$ correspondent à la moyenne pour chaque configuration. L'incertitude relative à la surface critique est aussi évaluée à partir des valeurs limites en chaque cas, laquelle est estimée à 5\% pour le diamètre $D=\SI{191}{mm}$, à 10\% pour les deux diamètres intermédiaires et à 15\% pour $D=\SI{76.2}{mm}$.

\begin{table}[H]
  \centering
  \caption{Conditions d'apparition du régime bidirectionnel utilisant les écarts types des thermocouples de la trémie.}
  %\setlength\extrarowheight{0.5pt}
  \begin{tabular}{ccccc}
    \hline
    $D\,(\si{mm})$ & $P_e\,(\si{kW})$ & $a_T\,(\si{cm^2})$ & $a^{crit}_B\,(\si{cm^2})$ & $(a_B/a_T)^{crit}\,(-)$ \rule{0pt}{14pt}\\[0.5ex]
    \hline
    76.2  & 1.2  &  45.6 &  5.5 & 0.12 \\
    76.2  & 2.0  &  45.6 &  4.7 & 0.10 \\
    76.2  & 4.0  &  45.6 &  3.9 & 0.09 \\
    \midrule[0.1pt]
    127.0 & 1.2  & 126.7 & 23.6 & 0.19 \\
    127.0 & 2.0  & 126.7 & 17.3 & 0.14 \\
    127.0 & 4.0  & 126.7 & 11.0 & 0.09 \\
    \midrule[0.1pt]
    152.4 & 2.0  & 182.4 & 31.4 & 0.17 \\
    152.4 & 4.0  & 182.4 & 28.3 & 0.16 \\
    \midrule[0.1pt]
    190.5 & 1.2  & 285.0 & 50.3 & 0.18 \\
    190.5 & 2.0  & 285.0 & 48.7 & 0.17 \\
    190.5 & 3.4  & 285.0 & 50.3 & 0.18 \\
    190.5 & 4.0  & 285.0 & 47.1 & 0.17 \\
    \hline
    \end{tabular}
  \label{tab:result_bidi_tc}
\end{table}

Comme le montre le tableau \ref{tab:result_bidi_tc}, la surface critique de transition diminue avec la diminution de la surface de l'évent et avec l'augmentation de la puissance de la source. Plus le compartiment est chaud plus il faut réduire la surface de l'amenée d'air en partie basse pour augmenter la dépression interne du local et déstabiliser l'écoulement monodirectionnel. On remarque aussi une augmentation du rapport $R=a_B/a_T$ critique avec l'augmentation de la surface de l'exutoire pour la même puissance. Cet constat est lié à la température dans le local au moment de la transition laquelle est supérieure pour les petits diamètres. Pour les petits diamètres, le débit d'extraction est plus faible et la pièce accumule plus d'énergie au cours de l'essai augmentant la température du local. Le rapport critique de $R$ pour la transition se situe dans la gamme ${[0.09,0.18]}$. Dans la section suivante ces résultats sont confrontés à ceux obtenus utilisant un critère visuel pour les mêmes essais réalisés.

La figure \ref{fig:ab_bidi_TC} illustre la relation entre la surface de l'exutoire et la surface critique d'amenée d'air $a_B^{crit}$.

\begin{figure}
\centering
\resizebox{0.7\textwidth}{!}{\input{figures/ab_bidi_at_TC_final-fig}}
\caption{Surface de transition $a^{crit}_B$ déterminée par le critère des écarts type en fonction de la surface de l'évent $a_T$.}
\label{fig:ab_bidi_TC}
\end{figure}

\section{Visualisation de l'écoulement}
Cette section décrit l'approche visuelle employée pour identifier la transition. Si les enceintes sont ensemencées, la différence entre un régime monodirectionnel et bidirectionnel est détectable à l’œil nu avec un plan laser. Bien que ce soit une approche qualitative, l'observation reste une méthode fiable dans certaines limites, et c'est pourquoi on a décidé de l'utiliser et la confronter aux résultats obtenus avec le critère de signaux de températures.

La visualisation de l'écoulement est réalisée avant d'atteindre l'état quasi-stationnaire dans le dispositif. Avec un laser portable les particules ensemencés dans le local supérieur sont éclairées au niveau du sol de ce compartiment pour observer l'écoulement montant, qui sort et ``coupe'' la masse d'air ensemencé autour de la trémie. L'écoulement qui descend est observé depuis le local inférieur éclairant la trémie.

La figure \ref{photo:mono} illustre un écoulement montant pas ensemencé vue de coté qui occupe toute la section de l'orifice. Si on éclaire cet écoulement avec un plan laser parallèle au sol du compartiment supérieur et on observe la trémie vue par dessus, on remarque que les particules qui montent depuis le compartiment inférieur forment une ``galette'' occupant l'aire de la trémie. 

\begin{figure}
\centering
\begin{tikzpicture}
  \node[anchor=south west,inner sep=0] (image) at (0,0) {\includegraphics[height=5cm]{Visu_mono.eps}};
  \begin{scope}[x={(image.south east)},y={(image.north west)}]
  \draw[white, -latex, line width=2.0pt] (0.45,0.21) -- (0.45,0.45);
  \draw[white, dashed, line width=1.0pt] (0.44,0.18) ellipse (0.18 and 0.15);
  \draw (0.14,0.17) node {\color{white} \textbf{Trémie}};
  \end{scope}
\end{tikzpicture}
\caption{Écoulement monodirectionnel montant à la sortie de la trémie. Vue au niveau du sol du compartiment supérieur.}
\label{photo:mono}
\end{figure}

Dès qu'on diminue la surface $a_B$, cette ``galette'' commence à ``vibrer'' indiquant le début de la transition jusqu'à perdre sa forme en régime bidirectionnel avec un mouvement chaotique des particules qui rentrent et sortent de l'orifice. Pour estimer le début du régime bidirectionnel, l'écoulement est observé depuis le compartiment inférieur en éclairant la trémie et le plafond de ce compartiment. Le régime bidirectionnel est défini dès que les premières bouffées d'air froid sont aperçues dans le compartiment de façon continue comme le montre la figure \ref{photo:bouffe}. La fréquence d'apparition de ces bouffées augmente avec le confinement.

\begin{figure}
\centering
\begin{tikzpicture}
  \node[anchor=south west,inner sep=0] (image) at (0,0) {\includegraphics[height=5cm]{Bouffe1}};
  \begin{scope}[x={(image.south east)},y={(image.north west)}]
  %\draw[white, dashed, line width=0.7pt, rotate around={35:(0.6,0.60)}] (0.6,0.60) ellipse (0.65cm and 1.5cm);
  \draw[white, -latex, line width=2.0pt] (0.58,0.68) -- (0.66,0.56);
  \draw[white, dashed, line width=1.0pt] (0.5,0.7) ellipse (0.46 and 0.17);
  \draw (0.5,0.93) node {\color{white} \textbf{Trémie}};
  \end{scope}
\end{tikzpicture}
\caption{Écoulement bidirectionnel vu depuis le compartiment inférieur. Bouffées d'air froid descendant par la trémie dans le compartiment inférieur.}
\label{photo:bouffe}
\end{figure}

Pendant la phase de transition des bouffées d'air entrent et sortent rapidement par la trémie sans franchir son épaisseur et que les thermocouples peuvent enregistrer. Ceci peut expliquer la différence entre les résultats du critère basé sur les réponses des thermocouples et les résultats basés sur l'observation des bouffées d'air qui rentrent dans le local.

Pour certains cas la différence entre les amplitudes et fréquences des fluctuations n'est pas significative pendant la phase de transition retardant le point de la détermination vers une condition plus confinée où le rapport $R$ est inférieur au celui identifié visuellement. Ces cas sont pour les deux diamètres plus petits dont les débits d'extraction des gaz chauds sont plus faibles même si les températures du local sont supérieures. Les résultats de l'estimation de l'apparition du régime bidirectionnel sont présentés dans le tableau \ref{tab:result_bidi_visu}.

La gamme du rapport critique $R^{crit}=(a_B/a_T)^{crit}$ pour les essais réalisés est entre 0.09 et 0.18. L'incertitude relative à la surface critique est aussi évaluée à partir des valeurs limites en chaque cas, laquelle est estimée à 5\% pour le diamètre $D=\SI{191}{mm}$, à 10\% pour les deux diamètres intermédiaires et à 15\% pour $D=\SI{76.2}{mm}$.

\begin{table}[H]
  \centering
  \caption{Conditions d'apparition du régime bidirectionnel utilisant le critère visuel.}
  %\setlength\extrarowheight{0.5pt}
  \begin{tabular}{ccccc}
    \hline
    $D\,(\si{mm})$ & $P_e\,(\si{kW})$ & $a_T\,(\si{cm^2})$ & $a^{crit}_B\,(\si{cm^2})$ & $(a_B/a_T)^{crit}\,(-)$ \rule{0pt}{14pt}\\[0.5ex]
    \hline
    76.2  & 1.2  &  45.6 &  5.5  & 0.12 \\
    76.2  & 2.0  &  45.6 &  4.7  & 0.10 \\
    76.2  & 4.0  &  45.6 &  3.9  & 0.09 \\
    \midrule[0.1pt]
    127.0 & 1.2  & 126.7 & 20.4  & 0.16 \\
    127.0 & 2.0  & 126.7 & 17.3  & 0.14 \\
    127.0 & 4.0  & 126.7 & 14.1  & 0.11 \\
    \midrule[0.1pt]
    152.4 & 2.0  & 182.4 & 26.7 & 0.15 \\
    152.4 & 4.0  & 182.4 & 23.6 & 0.13 \\
    \midrule[0.1pt]
    190.5 & 1.2  & 285.0 & 50.3 & 0.18 \\
    190.5 & 2.0  & 285.0 & 47.1 & 0.17 \\
    190.5 & 3.4  & 285.0 & 41.6 & 0.15 \\
    190.5 & 4.0  & 285.0 & 39.3 & 0.14 \\
    \hline
    \end{tabular}
  \label{tab:result_bidi_visu}
\end{table}

La figure \ref{fig:ab_bidi_visu} illustre graphiquement la relation entre la surface de l'exutoire et la surface critique définie quand les premières bouffées d'air ont été observées. 

\begin{figure}
\centering
\resizebox{0.7\textwidth}{!}{\input{figures/ab_bidi_at_visu_final-fig}}
\caption{Surface d'amenée d'air $a_B$ de transition déterminée visuellement en fonction de la surface de l'évent $a_T$.}
\label{fig:ab_bidi_visu}
\end{figure}

\section{Discussion}
Les résultats obtenus avec le critère visuel seront retenus par la suite de cette étude et utilisés à présent pour les confronter avec des résultats disponibles dans la littérature.

Les résultats expérimentaux sont d'abord comparés avec le modèle proposé par \textcite{hunt_coffey_2010} issu de leurs travaux sur le processus de vidange d'un fluide lourd contenu dans un local dans un milieu ambiant. Bien que ce soit une physique différente, ce modèle est proposé comme une extension du modèle de remplissage-vidange et peut nous fournir un ordre de grandeur de la section critique. \textcite{hunt_coffey_2010} proposent, suite à leurs expériences, un nombre de Froude critique $\Fr_{crit}=0.33$ pour estimer la transition de régime. Ce nombre de Froude est exprimé en termes de l'écoulement et associé aux paramètres géométriques du local et à la hauteur initiale de la couche du fluide $h_0$ plus dense que l'environnement.

Utilisant les surfaces critiques obtenues dans l'expression \eqref{eq:Fr_hunt} du nombre de Froude associé à la trémie avec les coefficients de débit $c_B=c_T=0.60$ et la hauteur $H$ du local comme la couche qui pilote l'écoulement pour le calcul de $\lambda_T$, on obtient les valeurs suivantes de $\Fr_{crit}$.

\begin{table}[!h]
  \centering
  \caption{Nombre de Froude à la trémie de la transition ($\Fr_{crit}$) basé sur l'expression \eqref{eq:Fr_hunt} proposée par \textcite{hunt_coffey_2010}.}
  \setlength\extrarowheight{0.5pt}
  \begin{tabular}{ccccc}
    \hline
    $a_T\,(\si{cm^2})$ & $a^{crit}_B\,(\si{cm^2})$ & $P_e\,(\si{kW})$ & $(a_B/a_T)^{crit}\,(-)$ & $\Fr_{crit}\,(-)$ \rule{0pt}{14pt}\\[0.5ex]
    \hline
     45.6 &  5.5 & 1.2 & 0.12 & 0.39 \\
     45.6 &  4.7 & 2.0 & 0.10 & 0.34 \\
     45.6 &  3.9 & 4.0 & 0.09 & 0.28 \\
    \midrule[0.1pt]
    126.7 & 18.8 & 1.2 & 0.16 & 0.40 \\
    126.7 & 15.7 & 2.0 & 0.14 & 0.34 \\
    126.7 & 12.6 & 4.0 & 0.11 & 0.28 \\
    \midrule[0.1pt]
    182.4 & 25.9 & 2.0 & 0.15 & 0.33 \\
    182.4 & 22.8 & 4.0 & 0.13 & 0.30 \\
    \midrule[0.1pt]
    285.0 & 50.3 & 1.2 & 0.18 & 0.36 \\
    285.0 & 46.3 & 2.0 & 0.17 & 0.34 \\
    285.0 & 40.8 & 3.4 & 0.15 & 0.30 \\
    285.0 & 35.3 & 4.0 & 0.14 & 0.28 \\
    \hline
    \end{tabular}
  \label{tab:fr_hunt_bidi}
\end{table}

Pour les puissances électriques de 2 à \SI{4}{kW} les valeurs de ce nombre de Froude sont en accord avec le modèle de \textcite{hunt_coffey_2010}. Ces résultats sont également sensibles au coefficient de débit qu'on ne peut pas quantifier avec nos expériences et qu'on a supposé constante. Cette hypothèse est valable pour des écoulements turbulents avec un nombre de Reynolds élevé \parencite{ward_1980} et typiquement estimé à une valeur de 0.60 pour les applications de ventilation naturelle des bâtiments \parencite{hunt_linden_2001}. Si on utilise par exemple une valeur de $c_B=c_T=0.70$, les valeurs du nombre du Froude augmentent de 17\% et seulement les résultats avec un puissance $P_e=\SI{4}{kW}$ sont en-dessous du seuil critique de $\Fr=0.33$. La dépendance du coefficient de débit sur le contraste de masse volumique a été quantifiée expérimentalement par \textcite{hunt_holford_2000} et \textcite{holford_hunt_2001} à travers une ouverture horizontale en ventilation par déplacement. Les auteurs ont mis en évidence une grande variation du coefficient avec un contraste de densité élevé et ont suggéré une dépendance avec un paramètre associé au type de panache produit par la source. \textcite{vauquelin_2017} à partir d'expériences ont proposé un modèle théorique pour exprimer la dépendance du coefficient de débit associé à un écoulement type panache développé à la sortie de l'exutoire en ventilation par déplacement dans le cadre général non-Boussinesq.

Dans la recherche d'une expression entre les surfaces d'ouvertures, celle-ci peut être représentée sous la forme d'une fonction en puissance $f(x)=ax^b$. Si on applique une loi de puissance aux données expérimentales, on obtient les lignes en pointillé sur la figure \ref{fig:ab_bidi_visu_fit}. Le coefficient pour les trois puissances électriques examinées est estimé à $b=5/4$.

\begin{figure}
\centering
\resizebox{0.67\textwidth}{!}{\input{figures/ab_bidi_at_visu_fit_final-fig}}
\caption{Surface d'amenée d'air $a_B$ critique de transition en fonction de la surface de l'évent $a_T$. Les lignes en pointillés représentent une possible loi en puissance.}
\label{fig:ab_bidi_visu_fit}
\end{figure}

D'autre part, l'écoulement d'échange en régime bidirectionnel examiné par \textcite{epstein_1988} montre la proportion $Q_{ex} \propto a^ {5/4} g'^{1/2}$, où $g'$ est la gravité réduite. Cela nous indique une dépendance de ce type d'écoulement avec la surface de l'ouverture horizontale et la flottabilité induite par la différence de masse volumique du local avec l'ambiant. Si on considère que cette dépendance se conserve jusqu'à la transition, au moins celle avec la surface de l'ouverture, la surface critique en partie basse que provoque un régime bidirectionnel jusqu'à la transition devrait être aussi en proportion avec la surface de l'exutoire en $a_T^{5/4}$.

La constante $a$ de cette proposition empirique qui associe la puissance de la source et par conséquence la température du local inférieur, a été déterminée par l'ajustement des courbes expérimentales. Les valeurs obtenues de la constante sont: $a=[0.044,0.040,0.034]$ pour les puissances électriques $P=[1.2,2.0,4.0]\,\si{kW}$ respectivement. Cependant dans ces essais la relation entre la puissance électrique et la puissance convective n'est pas connue et représente une limitation dans cette proposition. Une fonction de la puissance convective ($f(\dot{Q}_c))$ est plus appropriée qu'une constante $a$ de proportionnalité.

Pour estimer la relation de la température du local avec la puissance de la source sur la surface critique, le paramètre $\eta$ a été considéré et déterminé pour les différents essais réalisés. Le paramètre $\eta$ défini par $\eta=\Delta \rho/\rho = \Delta T /T_0$, représente la force motrice de la flottabilité exprimée par l'écart entre les fluides ambiant et léger. Ce paramètre intervient aussi dans l'expression du débit.

La température de la couche a été prise dans l'intervalle où la transition était observée et calculée comme la moyenne du profil vertical de température sur toute la hauteur. Dans l'annexe \ref{ch:couche} on présente les différents critères examinés pour l'estimer.

Le paramètre $\eta$ à la transition ($\eta^{crit}$) pour les différents essais est présenté sur la figure \ref{fig:eta_ab_at_visu}. Sur cette figure on illustre la gamme des températures de cette couche obtenues où le paramètre $\eta$ varie entre 0.07 et 0.32 qui correspond à une gamme de température entre 45 et \SI{120}{\celsius}. Cette information n'est pas suffisante pour déterminer la relation du flux de flottabilité avec la surface critique. D'autres mesures sont nécessaires pour généraliser la relation entre les surfaces d'ouvertures à la transition proposée.

\begin{figure}
\centering
\resizebox{0.67\textwidth}{!}{\input{figures/eta_tremie_ab_at_visu-fig}}
\caption{Rapport des ouvertures à la transition en fonction du déficit de masse volumique déterminée à partir de la température moyenne du local inférieur.}
\label{fig:eta_ab_at_visu}
\end{figure}

On remarque sur cette figure deux tendances principales avec pentes différentes. Une première pente plus raide pour une puissance électrique constante qui est associée à l'effet de la température. L'autre pente est associée à la section de l'ouvrant en reliant les points de chaque puissance de gauche à droite. Cette différence indique que la transition est plus sensible à la section de l'ouvrant horizontal que à la température de l'enceinte.

\section{Conclusions}

Ce chapitre est axé sur la détermination expérimentale de la condition d'apparition du régime bidirectionnel au niveau d'un ouvrant horizontal. Cette condition est issue de la réduction de la surface d'amenée d'air en partie basse d'un local ventilé naturellement avec une source de flottabilité constante au niveau du sol. Pour ces expériences, des résistances électriques ont été utilisées comme source de flottabilité en variant sa puissance avec un variateur de tension.

La première partie de cette étude a consisté à étudier l'influence de la surface en partie basse sur la réponse des thermocouples positionnées dans l'épaisseur de l'exutoire. Les mesures sont réalisées à partir d'un état quasi-stationnaire après la mise en chauffe du dispositif expérimental. Une méthode basée sur la règle de $3\sigma$ est proposée pour estimer la phase de transition en fonction des écarts types des températures et la fréquence des événements à franchir le seuil de trois écart type. Ce seuil est défini à partir de l'écart type base moyen déterminé pour un régime monodirectionnel avec une grande ouverture en partie basse. L'effet de la puissance est plus important que l'effet du diamètre de l'exutoire sur l'écart type et c'est le critère retenu par la suite. D'autre part, la variation des fluctuations dans la phase de transition par rapport au régime monodirectionnel n'est pas très évidente pour les deux diamètres plus petits examinés. Pour ces cas, la transition est décalée à une configuration plus confinée par rapport aux visualisations de l'écoulement dans les mêmes conditions.

Ces premiers résultats ont montré les effets du diamètre de l'exutoire et de la puissance de la source sur la surface critique à la transition. La réduction de la surface d'amenée d'air produit une dépression dans le local qu'à partir d'une certaine valeur déclenche le régime bidirectionnel au niveau de l'exutoire pour conserver le débit massique dans l'enceinte. La surface critique augmente avec la surface de l'exutoire dû à la diminution de la température du local. Cette diminution de la température est liée à l'augmentation du débit d'extraction lequel est proportionnel à la surface de l'exutoire. D'autre part, si on augmente la puissance de la source, le local devient plus chaud et on doit réduire la surface critique pour la même taille d'exutoire pour produire la dépression nécessaire et provoquer un écoulement bidirectionnel.

D'autres méthodes basées sur les réponses des thermocouples ont été testées pour estimer la transition. Ces méthodes comme la variation des températures moyennes de la trémie, la variation des écart types de ces températures moyennes ou la variation des moyennes des écart types, ne sont pas présentées dans ce mémoire parce que les résultats ne sont pas concluants pour estimer le début de la transition. L'effet de moyenner tous les signaux des thermocouples engendre une erreur d'appréciation sur le comportement de l'écoulement pendant la phase de transition et sur l'estimation de son apparition utilisant que les signaux. Pour la condition complétement confinée l'écoulement bidirectionnel observé par \textcite{varrall_2016} montre une organisation géométrique en moyenne et en conséquence les fluides ont des zones de préférence pour monter ou descendre à travers la trémie en moyenne. Dû à ce fait, on a supposé un comportement similaire jusqu'à la transition en partant du confinement total jusqu'à la transition. Par conséquent, l'analyse individuelle des thermocouples semble plus judicieuse et c'est celle qui est retenue par rapport aux autres méthodes mentionnées. On s'est focalisé sur les thermocouples les plus sensibles à un changement du confinement, malgré les faibles variations des fluctuations dans certains essais pour détecter la transition.

La seconde partie a été consacrée à la confrontation des résultats obtenus précédemment avec l'observation de l'écoulement intervalle par intervalle jusqu'à repérer les premières bouffées d'air descendant par l'orifice. Pour les essais où la variation des fluctuations montraient un changement appréciable, les deux critères coïncident en l'identification de la transition de régime. Les résultats obtenus visuellement sont retenus car l'existence d'un écoulement bidirectionnel a été confirmé par l'observation.

Une relation entre les surface d'ouvertures $a_B^{crit} = f(\dot{Q}_c)\, a_T^{5/4}$ est proposé basée sur les résultats obtenus expérimentalement. Cependant, la détermination de la fonction associée à la puissance convective ($f(\dot{Q}_c))$ nécessite une caractérisation expérimentale de la source et l'apport d'autres données comme les débits au niveau des ouvrants et le débit de flottabilité de la source sont nécessaires.


%\subsection{Étude dans la hauteur de la trémie}

%\subsection{Étude de la géométrie de l'amenée d'air (écart $\sim$ 10 \%)}

%\section{Condition d'apparition du régime bidirectionnel à la trémie}

%\subsection{Résistance électriques}

%\subsubsection{Effet de puissance}

%\subsubsection{Effet du diamètre}

%\subsubsection{Étude au sein de la trémie avec un peigne des TC à trois hauteurs}

%\subsubsection{\(\Delta\) P avec l'ambiant}


%%% Local Variables:
%%% mode: latex
%%% TeX-master: "../master"
%%% End:
